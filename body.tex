\section{Introduction}

This document provides an informal guide to the everyday mechanisms underpinning LSST Data Management's approach to project management.
It is intended to be read in conjunction with \citeds{LDM-472}, which provides a formal description of the project management process and requirements.

\section{Important Documents}
\label{sec:documents}

Wherever a conflict arises, baselined project documentation takes precedence over this note.
You are encouraged to submit bug reports so that this document can be made compliant.

Be aware of prefixes: ``LDM-'' documents refer specifically to the Data Management subsystem, ``LSE-'' to Systems Engineering, ``LPM-'' to Project Management.

\paragraph*{\citeds{LPM-43, LPM-44}}
\emph{\gls{wbs}} and \emph{\gls{wbs} Dictionary}, respectively.
The former shows the overall work breakdown structure for the whole project.
Note that these documents are periodically extracted from the master \gls{pmcs} system, and therefore occasionally do not reflect the most recent changes.

\paragraph*{\citeds{LPM-98}}
\emph{LSST Project Controls System Description}.
Describes and defines the components of the \gls{pmcs} used to manage and report on the overall LSST Project.

\paragraph*{\citeds{LDM-472}}
\emph{LSST DM Project Management and Tools}.
The formal, high-level document which defines the project management process used by LSST DM.
The present document may be thought of as a guide to applying the principles defined in \citeds{LDM-472} in practice.

In addition, you should be familiar with the \href{https://project.lsst.org/meetings/lsst2014/node/100}{EVMS Training -- CAM 101} slides presented by the Project Controls Specialist (\S\ref{sec:contacts}) at the \href{https://project.lsst.org/meetings/lsst2014/}{LSST 2014 Meeting}.

\section{Useful Contacts}
\label{sec:contacts}

The LSST DM Project Manager is William O'Mullane and the Deputy PM is John Swinbank.
They are the first point of contact for all issues regarding project management within DM.
The Subsystem Scientist is Mario Juric who is the first point of contact for science or scientist related questions.

The LSST Project Controls Specialist is Kevin Long.
He is responsible for the \gls{pmcs} and, in particular, for ensuring that DM properly complies with our earned value management requirements.
He is the first point of contact for all questions regarding \gls{pmcs}.

\section{Technical Managers}
\label{sec:tcam}

This guide is primarily aimed at the LSST DM technical managers.
Technical managers report directly to the DM Project Manager.
Technical managers are, in general, expected to act as \gls{cam} and technical lead for their \gls{wbs} \glspl{element}; as such, they are sometimes referred to as ``T/CAMs''.

The T/CAM role is described in \citeds{ldm-294}.
The role of \gls{cam} is defined in detail in \S\ref{sec:structure}.

\section{Formal Organizational Structure}
\label{sec:structure}

\subsection{Work Breakdown Structure}
\label{sec:wbs}

The LSST \gls{wbs} is defined in \citeds{LPM-43} (see also \citeds{LPM-44} for an extended---but not universally illuminating---definition of what each level of the breakdown consists of).

The \gls{wbs} provides a hierarchical index of all hardware, software, services, and other deliverables which are required to complete the LSST Project.
It consists of alphanumeric strings separated by periods.
The first component is always ``1'', referring the LSST Construction Project.
``02C'' in the second component corresponds to Data Management Construction.
Subdivisions thereof are indicated by further digits.
Subdivisions at this level correspond to teams within the DM project.
Thus:

\begin{longtable}[]{@{}lll@{}}
\hline
\gls{wbs} & Description & Lead Institution\tabularnewline
\hline
\endhead
1.02C.01 & System Management & LSST\tabularnewline
1.02C.02 & Systems Engineering & LSST\tabularnewline
1.02C.03 & Alert Production & University of Washington\tabularnewline
1.02C.04 & Data Release Production & Princeton University\tabularnewline
1.02C.05 & Science User Interface & Caltech IPAC\tabularnewline
1.02C.06 & Science Data Archive & SLAC\tabularnewline
1.02C.07 & Processing Control \& Site Infrastructure &
NCSA\tabularnewline
1.02C.08 & International Communications. \& Base Site & NCSA \&
LSST\tabularnewline
1.02C.09 & Systems Integration \& Test & LSST\tabularnewline
1.02C.10 & Science Quality \& Reliability Engineering &
LSST\tabularnewline
\hline
\end{longtable}

These subdivisions are referred to as the \emph{third level \gls{wbs}}.
Often, they are quoted without the leading ``1'' (e.g. ``02C.01''), but, even in this form, they are referred to as ``third level''.

All of these third level \gls{wbs} \glspl{element} are subdivided, forming a fourth level.
The fourth level always contains a ``00'' \gls{element}, which is used to capture management and \gls{loe} work, and may contain other fourth level, or even deeper, structure.
Nodes in the \gls{wbs} tree are referred to as \glspl{element}.

\subsection{Organization Breakdown
Structure}\label{organization-breakdown-structure}

In parallel with the \gls{wbs}, we have an \gls{obs}, which assigns each institution involved in the project a unique numeric identifier.
The \gls{obs} is defined in \citeds{LPM-98}.
Those institutions directly relevant to DM include:

\begin{longtable}[]{@{}ll@{}}
\hline
\gls{obs} & Institution\tabularnewline
\hline
\endhead
1.01 & LSST\tabularnewline
1.02 & SLAC\tabularnewline
1.03 & Caltech IPAC\tabularnewline
1.04 & NCSA\tabularnewline
1.05 & University of Washington\tabularnewline
1.06 & Princeton University\tabularnewline
\hline
\end{longtable}

\subsection{The Control Account
Manager}\label{the-control-account-manager}

A \gls{ca} is the intersection between the \gls{wbs} and the \gls{obs}.
Each \gls{ca} falls under the purview of a \gls{cam}.
Typically within DM, a single \gls{cam} is responsible for the whole of a third level \gls{wbs}.
That is, the manager at the lead institution for a particular component is responsible for all work performed on that \gls{wbs} \gls{element}, even if some of that work is performed at another institution.

\section{Earned Value Principles}
\label{sec:evms}

LSST DM is funded by as an \gls{nsf} \gls{mrefc} project.
Under the terms of the \gls{mrefc} award, we are required to follow an \emph{earned value} approach to project management.
A full description of the earned value approach is outside the scope of this document: the project will provide formal training.
We provide a brief aide-m\'emoire for convenience only.

The earned value technique assigns each component of the system with a
dollar value corresponding to its expected cost of production. In a
(largely) software based project like LSST DM, it is often convenient to
equate the cost of production with the cost of the labor required to
write the code: in the more general case, however, it also includes cost
of hardware procurements, etc. This provides a convenient heuristic for
estimating cost: given some nominal labor costs, the cost of a component
is a proxy for the amount of labor required to produce it.

As well as a cost, the plan includes a start date and a completion date
for each component.

The total value of work which \emph{should} have been completed by a particular date is the \gls{bcws}.
The total value of work which has \emph{actually} been completed by the date is the \gls{bcwp}.
The total sum expended on the work is the \gls{acwp}.
Theoretically, if estimates of both cost and time for every component of the system are accurate, at the end of construction, all of these three quantities will be equal.

In practice, estimation is rarely perfect.
Imperfect estimates are exposed as variances.
Specifically, we can show either \gls{sv}---a negative value means that less
of the system has been delivered to date than planned---or \gls{cv}---a negative value means that the work delivered to date has been more expensive than predicted.
Related quantities, \gls{spi} and \gls{cpi}, express the same information as ratios rather than sums.
In general, we strive to achieve variances of near zero: even a positive variance (corresponding to being ahead of schedule or being cheaper than expected) is indicative of an inaccurate plan.

All of these indices can be applied to any \gls{wbs} \gls{element} within the project.
Thus, we can talk about value earned across the whole of DM (1.02C) or on a specific component (say, the User Workspace Toolkit, 1.02C.05.05).

\subsection{Labor Costs}
\label{sec:labor-costs}

Our methodology is designed to avoid exposing individual salaries to the wider project.
Therefore, when calculating labor costs for earned value purposes, we do not rely on a known cost per individual.
Instead, all staff are assigned to one of a number of types (typically within DM we use scientist, senior scientist, developer or senior developer, but there are a several alternatives available at the project level: see \citeds[LPM-81 table 5-2]{LPM-81} for the full list), each of which is assigned a nominal cost level according to institution: it does not vary between individuals of the same type within the same institution.
This nominal cost does not, therefore, correspond to a particular individual, but is a broadly defined expectation.
Full details are available in \citeds{LPM-81}.

\subsection{Variance Narratives}
\label{sec:variance-narrative}

Every month, the \gls{ecam} tool is updated from the \gls{pmcs} to reflect the latest earned value status.
If either cost or schedule is behind schedule by more than either \$100,000 or 10\% you are required to provide a ``narrative''.
This is divided into two parts: you must explain why the variance arose, and what action will be taken to correct it (e.g.  slipping work into the future, or diverting resources from elsewhere to make up the shortfall).
The narrative is entered directly into \gls{ecam}.

In future, narratives may also be required for positive variances (i.e. running ahead of schedule).

Variance is calculated on a monthly basis; variance narratives are due in the second week of the calendar month following that to which they apply (refer to \S\ref{sec:reporting-cycle} for details).

Variance is perfectly normal in a project and we should not be afraid to have them and provide narratives about them. We should take care the narrative is not always the same i.e. if we have a negative variance every month and it is because we did not plan for something we should do better planning.

\subsection{Level of Effort Work}
\label{sec:loe}

The implicit assumption in the earned value technique outlined above is that all work corresponds to a specific deliverable.
However, parts of our work do not: every member of the team will find it necessary to attend meetings or take part in other activities which do not directly map to deployed code.
This may be particularly the case for technical managers or others in leadership roles within the project.
This work is referred to as \gls{loe}: it is assumed to earn value simply through the passage of time.

\citeds{LDM-472} provides a detailed definition of the types of work it is permissible to regard as \gls{loe}.
In general, we strive to minimize the fraction of our effort which is devoted to \gls{loe} activities and favor those which are more directly accountable.
In certain cases such as operations of pipelines or other systems \gls{loe} is perfectly acceptable.

The assumption encoded in \citeds{LDM-472} is that developers will spent 30\% of their time on \gls{loe} type activities, and the remaining 70\% of their effort is tracked against concrete deliverables.
This does not have to be used as a rule however, if we can plan activities to more than 70\% of the time we should do so.

\section{Estimating Effort}
\label{sec:effort}

The Project assumes that a full-time individual works for a total of
1,800 hours per year: this figure is \emph{after} all vacations, sick
leave, etc are taken into account. Staff appointed to ``developer''
positions are expected to devote this effort directly to LSST.

Appointment as a ``scientist'' includes a 20\% personal research time
allowance. That is, scientists are expected to devote 1,440 hours per
year to LSST, and the remainder of their time to personal research.

Personal research time is \emph{not} chargeable to LSST under any \gls{wbs} or
account, including level of effort. The Project expects to pay the full
rate for an individual with research time who contributes 1,440 hours to
the project, and does not require any accounting of the remaining 360
hours.

When reporting actual costs (\S\ref{sec:actuals}), it may
be helpful to consider the following examples:

A developer for which the total annual cost (salary, overheads, etc) is
\(A\) charges an hourly rate of \(A / 1800\).

A scientist with total annual cost \(B\) charges an hourly rate of
\(B / 1440\).

No further corrections are necessary. In particular, there is no
difference in the way working hours are measured, or the conversion of
\glspl{sp} to hours.

Some individuals serve ``Science Lead'' (SL) roles within DM which ,
such as the Project Scientist and Pipelines Scientist. These roles are
not equivalent to being granted personal research time, but reflect a
level of scientific oversight within the project. Time spent performing
this role must be accounted for in the usual way (either as \gls{loe} or as
providing deliverables), and charged to an account agreed with the
DM Project Manager (\S\ref{sec:contacts}). They generally
serve as the Product Owners for parts of the system their respective
institutions have been tasked to deliver (not all products, as we
discussed). While SLs report to the Subsystem Scientist, they primarily
provide a service to the local T/CAM.

Science leads are typically not 100\% and as a rough guide we consider
they spend about half their allocated time serving the Subsystem
Scientist and half serving the T/CAM. As much work as possible should be
accounted for in stories.

Our base assumption is that 30\% of an individual's LSST time (i.e. 540 hours/year for a developer, 432 hours/year for a scientist) are devoted to overhead for meetings, ad-hoc discussions and other interruptions.
This work is counted as \gls{loe} (and, as such, is charged to the relevant ``00'' fourth level \gls{wbs} \gls{element}, as described in \S\ref{sec:loe}).
However meeting attendance is well understood:  on the planning side we should allocate stories and points for individuals attending and preparing for meetings.
If we fail to correctly visualize this work we loose track of it.
If we bill less to \gls{loe} and have to fill a variance narrative that is fine.

Some individuals---notably technical managers themselves, as well as
others in leadership roles---are expected to have a larger fraction of
their time devoted to \gls{loe} work.

Assuming no variation throughout the year, we therefore expect 105 hours
of productive work from a developer, or 84 hours from a scientist, per
month. Note that this is averaged across the year: some months, such as
those containing major holidays, will naturally involve less working
time than others: the remainder will necessarily include more working
time to compensate.

Rather than working in hours, our \gls{jira} based system uses \glspl{sp}, with one \gls{sp} being defined as equivalent to four hours of effort by a competent developer.
Thus, we expect developers and scientists to produce 26.25 and 21 \glspl{sp} per \emph{average} month respectively.
This is summarized in Table \ref{tab:working-rate}.

\begin{table}
\begin{longtable}[]{@{}lrrr@{}}
\hline
          & \multicolumn{2}{c}{Hours} & \multicolumn{1}{c}{\glspl{sp}} \\
          & Per year & Per month      & Per month \\
\hline
Developer & 1800     & 105            & 26.25 \\
Scientist & 1440     &  84            & 21.00 \\
\hline
\end{longtable}
\caption{Expected working rates for developers and scientists.}
\label{tab:working-rate}
\end{table}

On occasion, it may be appropriate to tailor the number of \glspl{sp} expected per unit time from a particular individual.
For example:

\begin{itemize}
\item
  Individuals in leadership roles may assign a larger fraction of their time to \gls{loe} type work, and therefore spend fewer hours generating \glspl{sp}.
  The ratio of hours to \glspl{sp} remains constant, but the number of hours decreases.
\item
  New or inexperienced developers, even when devoting their full attention to story-pointed work, will likely be less productive than their more experienced peers.
  In this case, the ratio of hours to \glspl{sp} increases, but the number of hours remains constant.
\end{itemize}

In either case, the total number of \glspl{sp} which will will be generated by the team in a given time interval is reduced.
This should be taken into account when \gls{resloading}.

For every \gls{story} we should record the points we actually spent on the \gls{story} versus the planned points.
We must all monitor this as individuals to improve our planning performance.

\section{Long Term Planning}
\label{sec:long-term-plan}

Refer to \citeds{LDM-472} for a description of the long-term planning system.
In brief, the plan for the duration of construction is embodied in:

\begin{enumerate}
\item
  A series of \emph{planning packages}, which describe major pieces of
  technical work. Planning packages are associated with concrete, albeit
  high-level, deliverables (in the shape of milestones, below), and have
  specific resource loads (staff assignments), start dates, and
  durations. The entire DM system is covered by around 100 of these
  planning packages.
\item
  \emph{Milestones} represent the delivery or availability of specific
  functionality. Each planning package culminates in a milestone, and
  may contain other milestones describing intermediate results.
\end{enumerate}

Planning packages are defined at the fourth level of the \gls{wbs} breakdown (e.g. at 1.02C.04.02; see \S\ref{sec:wbs}).
They may not cut across the \gls{wbs} structure, but rather must refer to that particular fourth-level \gls{element} and its children.

Milestones are defined at a number of levels: see \citeds{LDM-472} for details.
To summarize:

\begin{description}
\item[Level 1]
These are chosen by the \gls{nsf} from a list defined by the Project.
\item[Level 2]
These reflect cross-subsystem commitments. As such, they must be defined
in consultation with the DM Project Manager.
\item[Level 3]
These reflect cross-third-level \gls{wbs} commitments. As such, they must be
defined in discussion between two or more technical managers.
\item[Level 4]
These are internal to a particular third-level \gls{wbs}, and can therefore be
specified by a single technical manager.
\end{description}

Some of these are exposed to external reviewers: it is vital that these
be delivered on time and to specification. Low-level milestones are
defined for use within DM, but even here properly adhering to the plan
is vital: your colleagues in other teams will use these milestones to
align their schedules with yours, so they rely on you to be accurate.

Relationships may be defined between milestones and between milestones
and planning packages. Often, as described in \citeds{LDM-472}, these are blocking
relationships: a particular activity cannot proceed until all the work
which blocks it has been completed. It is also possible to identify a
non-specific relationship between activities. This should be taken to
mean that they share some common aspects and hence it may be appropriate
to consider them together.

\subsection{Planning Research Work}
\label{sec:long-term-research}

In order for the DM system to reach its science goals, new algorithmic or engineering approaches must sometimes be researched.
It is appropriate to budget time for this research work in planning packages.
However, areas where successful delivery of the DM system is dependent on speculative research are a source of \gls{risk}: wherever possible, the plan should also provide for a fallback option to be taken when research objectives are not achieved.
When fallback options are not available, discuss how to account for this \gls{risk} with the DM Project Manager (\S\ref{sec:contacts}).

\subsection{Earned Value and Planning Packages}
\label{sec:long-term-value}

A planning package has a duration and a staff assignment (it is ``resource loaded'').
Given a (nominal) cost per unit time of the staff involved (see \S\ref{sec:labor-costs}), this translates directly to a \gls{bcws}.

During the \gls{cycle} planning process (\S\ref{sec:cycle-plan}), effort is drawn from the budget embodied in the planning packages to generate the \gls{cycle} plan, described in terms of \glspl{epic}: see \S\ref{sec:planning-epics} for details.
Each \gls{epic} itself has a particular budget.
This budget is subtracted from that available in the planning package at the point when the \gls{epic} is defined.

At any given time, the \gls{bcwp} of a planning package consists of the sum of the \gls{bcwp} of all \glspl{epic} derived from that package which have been marked complete, together with the fractions of value earned from all \glspl{epic} currently in progress.

An example may serve to illustrate.

Planning package \(P\) is baselined to start at the beginning of F17 and run through to the end of F18, i.e. a total of three \glspl{cycle}, or 18 months. It has two members of staff---\(A\) and \(B\)---assigned to it full time.
Both share the same nominal cost of \$\(X\) per \gls{cycle}.

The \gls{bcws} for the total planning package is the cost per \gls{cycle} multiplied by the number of \glspl{cycle}: \(3 \times 2 \times \$X = \$6X\).

In F17, both members of staff are assigned to six-month \gls{epic} derived from \(P\).
The \gls{bcws} of the \gls{epic} is \(\$2X\).
The remaining value in the planning package is \(\$4X\).

At the end of F17, the \gls{epic} is completed.
The \gls{bcwp} and \gls{acwp} are both \(\$2X\).
The work is on cost and on schedule: there is no variance.

In S18, \(A\) is reassigned and is unable to work on a new \gls{epic} derived from \(P\).
\(B\) continues the work alone, completing an \gls{epic} worth \(\$X\) by the end of the \gls{cycle}.
The \gls{bcwp} and \gls{acwp} are now both \(\$3X\); there is no cost variance.
However, the \gls{bcws} is \(\$4X\): compared to the original schedule for the planning package, there is a schedule variance of \(-\$X\).
There is a total of \(\$3K\) left in the planning package.

In F18, \(C\) joins the project.
\(C\) only costs \(\$0.5X\) per \gls{cycle}, but is a fast worker: she can complete in one \gls{cycle} work that would take \(A\) or \(B\) two \glspl{cycle}.

\(B\) and \(C\) work together through F18.
The \gls{acwp} for the \gls{cycle} is \(\$1.5X\); the \gls{bcwp} is \(\$3X\). The \gls{acwp} to date \(\$4.5X\).
The \gls{bcwp} and \gls{bcws} are both \(\$6X\).
At this point, the project is complete: there is no schedule variance, and a cost variance of \(+\$1.5X\).

\subsection{Epic-Based Long Term Plans}

As per \S\ref{sec:cycle-plan}, the \gls{epic} is the standard level of granularity for planning work over the relatively short term (periods of several months).
However, \glspl{epic} may also be valuable for longer-term, fine-grained planning.
When a detailed description of work for a given planning package is known, it can and should be described in \gls{jira} through a series of \glspl{epic} assigned to the appropriate \glspl{cycle}.
As long as they have not been scheduled for the current \gls{cycle}, these \glspl{epic} can be freely created and changed at any time, without any sort of approval process.
Of course, for this process to be practically useful, these \glspl{epic} should fit within the scope and budget of the relevant planning package.

Fine grained planning of this sort can be useful for ``bottom-up''
analysis of the work to be performed and validation of the resources
needed to implement a particular planning package. Thinking through the
plan in this way can help in building up a detailed plan in a flexible,
agile way, while also ensuring that scope, cost and schedule are
carefully controlled.

\subsection{Software Releases}\label{software-releases}

Per \S\ref{sec:cycle-cadence}, a series of software releases will be made
throughout LSST construction. These will provide a stable basis upon
which external users (other subsystems, science collaborations and the
wider community) can base their work.

Our releases follow a strictly time-based cadence. That is, they are
made on a pre-defined schedule which tracks our
short-term plan (\S\ref{sec:cycle-plan}), rather than
being guaranteed to provide a particular set of functionality. For this
reason, individual releases will not be exposed as milestones above
level 3 (exposing them at level 3 or below for internal use is
optional). Other parts of the project which depend on certain functions
being available should depend on a milestone describing that function,
rather than on a particular release of the software.

In addition to this cyclical official release process, we may provide
packaged distributions of the codebase at more frequent intervals in
support of commissioning or other activities that require a higher
release cadence or timely delivery of particular features.

\section{Short Term Planning}
\label{sec:cycle-plan}

Per \citeds{LDM-472}, short term planning is carried out in blocks referred to as \glspl{cycle}, which (usually) last for six months.
Before the start of a \gls{cycle}, technical managers work with the DM Project Manager and the Project Controls Specialist to ensure their plan for the \gls{cycle} is well defined in both \gls{jira} and the \gls{pmcs}.

\subsection{Cycle Cadence \& Release Planning}
\label{sec:cycle-cadence}

At the end of a \gls{cycle}, a release manager appointed from within the \gls{square} team will coordinate a public release of the codebase.
This release will consist of a coherent, well tested set of packages, together with release notes, documentation and performance characterization.

In order to make this possible, the release will be tagged two weeks before the end of the \gls{cycle}.
All work which is destined for the release must have been merged to the \texttt{master} branch by this point.
For the remainder of the \gls{cycle}, the priority is to provide bug fixes, documentation and other material in support of the release as requested by \gls{square}.
In so far as it does not interfere with that priority, other work may continue as normal, with the caveat that new development will not be included in a release until the end of the subsequent \gls{cycle}.

Throughout this process, the \gls{square} technical manager will advertise the
current state of the release to all interested parties using the
\href{https://community.lsst.org/}{LSST Community Forum}.

Technical managers of the other groups are responsible for providing to \gls{square} such material as is required to support the release.
This will include a set of release notes which provide a summary of work performed over the course of the \gls{cycle}.
Please liaise with \gls{square} in advance to establish the appropriate format and granularity of these notes.

\subsection{Defining The Plan}\label{defining-the-plan}

\subsubsection{Scoping Work}\label{scoping-work}

The first essential step of developing the short term plan is to produce an outline of the programme of work to be executed.
In general, this should flow directly from the long term plan (\S\ref{sec:long-term-plan}), ensuring that the expected planning packages are being worked on and milestones being hit.

While developing the \gls{cycle}, please:

\begin{itemize}
\item
  Do not add \emph{artificial} padding or buffers to make the schedule look good;
\item
  Do budget appropriate time for handling bugs and emergent issues;
\item
  Reserve time for planning the following \gls{cycle}: it will have to be defined before this \gls{cycle} is complete;
\item
  Leave time for other necessary activities, such as cross-team collaboration meetings and writing documentation.
\item
  Per the \gls{cycle} cadence (\S\ref{sec:cycle-cadence}), ensure that new development will conclude (or, at a minimum, be in a releasable state) in time for the end of \gls{cycle} release.
\end{itemize}

Obviously, ensure that the programme of work being developed is achievable by your team in the time available: ultimately, you will want to compare the number of \glspl{sp} your team is able to deliver (\S\ref{sec:effort}) with the sum of the \glspl{sp} in the \glspl{epic} you have scheduled (\S\ref{sec:planning-epics}), while also considering the skills and availability of your team.
It is better to under-commit and over-deliver than vice-versa, but, ideally, aim to estimate accurately.

\subsubsection{Defining Epics}
\label{sec:planning-epics}

As described in \citeds{LDM-472}, the plan for a six month \gls{cycle} fundamentally consists of a set of resource loaded \glspl{epic} defined in \gls{jira}.
Each \gls{epic} loaded into the plan must have:

\begin{itemize}
\item
  A concrete, well defined deliverable \emph{or} be clearly described as a ``bucket'' (\S\ref{sec:bucket});
\item
  The \texttt{cycle} field set to the appropriate \gls{cycle};
\item
  The \texttt{wbs} field set to the appropriate \gls{wbs} \emph{leaf} \gls{element}.
\item
  The \texttt{Story Points} field set to a (non-zero!) estimate of the effort required to complete the \gls{epic} in terms of \glspl{sp} (see \S\ref{sec:effort}).
\end{itemize}

Be aware that:

\begin{itemize}
\item
  An \gls{epic} may only be assigned to a single \gls{cycle}.
  It is not possible to define an \gls{epic} that crosses the \gls{cycle} boundary (see \S\ref{sec:cycle-close} for the procedure when an \gls{epic} is not complete by the end of the \gls{cycle}).
\item
  An \gls{epic} may only be assigned to a single \gls{wbs} leaf \gls{element}.
  It is not possible to define \glspl{epic} that cover multiple \gls{wbs} \glspl{element}.
  See \S\ref{sec:cross-team} for information on scheduling work which requires resources from multiple \glspl{element}.
\item
  An \gls{epic} must descend from a single planning package (see \S\ref{sec:long-term-plan}).
\item
  Although \gls{loe} work should be charged to the 00 fourth-level \gls{element} (\S\ref{sec:loe}), this does not imply that other work cannot be charged here.
  Indeed, where possible management activities \emph{should} be scheduled as \glspl{epic} with concrete deliverables in this \gls{element} rather than being handled as \gls{loe}.
\item
  The \gls{epic} should be at an appropriate level of granularity.
  While short \glspl{epic} (a few \glspl{sp}) may be suitable for some activities, in general \glspl{epic} will describe a few months of developer-time.
  \Glspl{epic} allocated multiple hundreds of \gls{story} points are likely too broad to be accurately estimated.
\end{itemize}

The Project Controls Specialist (\S\ref{sec:contacts}) will periodically (per \S\ref{sec:reporting-cycle}) pull information from \gls{jira} to populate \gls{pmcs} with the plan.

All \glspl{epic} which have \gls{wbs} and \gls{cycle} defined will be loaded into \gls{pmcs} (and must, therefore, have concrete deliverables and plausible \gls{sp} estimates).
\Glspl{epic} which do not satisfy these criteria may be defined in \gls{jira}.
These will not be pulled into \gls{pmcs}, will not form part of the scheduled plan, and will not earn value.
However, they may still be useful for organizing other work, sketching plans for future \glspl{cycle}, etc: please define them as necessary.

In order to fully describe the plan to \gls{pmcs}, \glspl{epic} require information that is not captured in \gls{jira}.
Specifically, it is necessary to define:

\begin{itemize}
\item
  Start and end dates for the \gls{epic};
\item
  Staff assignments.
\end{itemize}

Although it is possible---indeed, encouraged---to set the \texttt{assignee} field in \gls{jira} to the individual who is expected to carry out the bulk of the work in an \gls{epic}, this does not provide sufficient granularity for those cases when more than one person will be contributing.

In fact, it is only required to provide a staff assignment in terms of ``resource types'' (i.e. scientists, senior scientists, developers, senior developers, etc).
In practice, to ensure your team is evenly loaded, it is usually necessary to break it down to named individuals.

This information is most conveniently captured in per-team spreadsheets which are supplied to the Project Controls Specialist before the start of the \gls{cycle}.
Spreadsheets describing previous \glspl{cycle} are stored in \href{https://drive.google.com/drive/u/0/folders/0BxgFbTQURmr6TmxXSm5Dc1JJWk0}{Google Drive}: a convenient way to get started would be to use one of those as a template.

The spreadsheets used capture \gls{epic} start and end dates at monthly granularity.
This can lead to a variance (see \S\ref{sec:evms}) when monthly results are tabulated (it assumes that work for an \gls{epic} is evenly distributed across all the months in which it is scheduled).
In practice, this variance is likely to be small, and should average out by the end of the \gls{cycle}, when all \glspl{epic} should be closed on schedule.
However, if this becomes a problem, it is possible to fine-tune dates by directly consulting with the Project Controls Specialist.

When loading \glspl{epic} at the start of a \gls{cycle}, it is not necessary that they be fully loaded with \glspl{story} (defined as per \S\ref{sec:defining-stories}): these can be defined during the \gls{cycle}.
You do, of course, need to have thought through the contents of the \gls{epic} in enough detail to provide an overall \gls{sp} estimate and deliverables, though.

With the agreement of the Project Manager and Project Controls Specialist, it is acceptable to load the plan for a \gls{cycle} in three month ``chunks''.
That is, the plan for the first three months of the \gls{cycle} is loaded before the start of the \gls{cycle}, and the remaining part of the plan covering the final three months is loaded before the start of the fourth month.
This approach provides an opportunity to fine-tune the plan for the second half of the \gls{cycle}, without requiring a formal \gls{lcr} (\S\ref{sec:cycle-change}).

\subsubsection{Scheduling Research Work}\label{scheduling-research-work}

As discussed in \S\ref{sec:long-term-research}, research is sometimes required
to meet our objectives. However, it is not a natural fit to our usual
planning process, as it is speculative in its nature: it is often
impossible to produce a series of logical steps that will lead to the
required result. We acknowledge, therefore, that scheduling an \gls{epic} to
deliver some particular new algorithm based on the results of research
is impossible: we cannot predict with any confidence when the
breakthrough will occur.

We therefore schedule research in \gls{timebox}ed \glspl{epic}: we allocate a certain amount of time based on the resources available, rather than on an estimate of time to completion.
However, note that these \gls{timebox}ed \glspl{epic} should still provide concrete deliverables: they are not open-ended ``buckets'' as discussed elsewhere.
Since we cannot rely on the successful completion of the research project as a deliverable, we instead require that a summary of the research completed to date be delivered at the completion of the time allocated.
The presentation and format of this report will vary depending on the nature of the research (a \href{https://sqr-000.lsst.io/}{technical note} is a likely option), and, as usual (\S\ref{sec:planning-epics}), should be defined before the \gls{epic} is ingested to \gls{pmcs}.

\subsubsection{Bucket Epics}
\label{sec:bucket}

Some work is ``emergent'': we can predict in advance that it will be necessary, but we cannot predict exactly what form it will take.
The typical example of this is fixing bugs: we can reasonably assume that bugs will be discovered in the codebase and will need to be addressed, but we cannot predict in advance what those bugs will be.

This can be included in the schedule by defining a ``bucket'' \gls{epic} in which stories can be created when necessary during the course of a \gls{cycle}.
Make clear in the description of the \gls{epic} that this is its intended purpose: every \gls{epic} should either have a concrete deliverable or be a bucket.

Bucket \glspl{epic} have some similarities with \gls{loe} work.
As such, we acknowledge that they are necessary, but seek to minimize the fraction of our resources assigned to them.
If more than a relatively small fraction of the work for a \gls{cycle} is assigned to bucket \glspl{epic}, please consider whether this is really necessary and appropriate.

Be aware that even bucket \glspl{epic} must be assigned to a specific \emph{leaf} \gls{element} of the \gls{wbs}.
That is, it is not in general possible to define an \gls{epic} which handles bug reports or emergent feature requests across the whole of the codebase unless a specific \gls{wbs} leaf \gls{element} is devoted to maintenance activities of this type.
Instead, it may be necessary to define a different bucket \gls{epic} for each leaf of the \gls{wbs} tree.

\subsubsection{Mapping SPs to BCWS}
\label{sec:sps-to-bcws}

As discussed above, the amount of work to be performed is estimated in terms of \glspl{sp} (\S\ref{sec:effort}), while the earned value (\S\ref{sec:evms}) system accounts for work in terms of budgeted cost (\gls{bcws}).
In order to estimate the value earned by completing an \gls{epic}, it is necessary to map from one to the other.

The outline of the calculation here is straightforward: \glspl{sp} map to developer hours.
Given the staff assignment for the \gls{epic} (see \S\ref{sec:planning-epics}), the number of hours scheduled per developer can be calculated.
Given the nominal costs (per \S\ref{sec:labor-costs}) associated with each developer, the total labor cost can be estimated.

Therefore, we calculate the number of hours of each staffing grade being
assigned to the \gls{epic}, multiply that by the cost per hour of that grade,
and that provides the cost of the work scheduled.

\subsubsection{Cross Team Work}
\label{sec:cross-team}

Planning \glspl{epic} are always assigned to a particular \gls{wbs} leaf \gls{element}: they do not span \glspl{element} or teams.
It is therefore impossible to schedule a single \gls{epic} which covers cross-team work.
There are two ways to approach this problem:

\begin{itemize}
\item
  The technical managers for both teams to be involved in the work schedule \glspl{epic} separately, within their own \gls{wbs} structure.
  They are responsible for agreeing start and end dates, deliverables and resourcing between themselves.
  From the point of view of the \gls{pmcs}, these \glspl{epic} are independent pieces of work which happen to be
  coincident.
\item
  With agreement between technical managers, an individual may be
  detached from one team and explicitly work for another team for some
  defined period. One technical manager is therefore responsible for
  defining and scheduling their work. Their ``home'' manager will charge
  actuals (\S\ref{sec:actuals} against the \gls{wbs} supplied
  by the manager manager of the receiving team.
\end{itemize}

Regardless of the approach taken, technical managers should be
especially careful to ensure that cross-team work is well defined.
Usually, it is convenient for a single manager to take ultimate
responsibility for ensuring that it is successfully delivered.

\subsection{Revising the Plan}
\label{sec:cycle-change}

During the \gls{cycle}, it is possible that changing circumstances will cause reality not to exactly match with the plan.
This will ultimately cause a variance (see \S\ref{sec:evms}), which should be minimized and which---if it becomes significant enough---will require a narrative.

After the plan for the \gls{cycle} has been entered into \gls{jira}, it is under change control: it can only be altered through a \gls{lcr} approved by the \gls{ccb}.
In order to reschedule (or remove entirely from the \gls{cycle}) an \gls{epic} which has not yet started, the technical manager should work with the Project Controls Specialist (\S\ref{sec:contacts}) to prepare and submit an appropriate \gls{lcr} to the \gls{ccb}.
The \gls{ccb} meets on the third Wednesday of the calendar month; change requests must be submitted well in advance of this.
Therefore, it is advisable to take time early in the calendar month to review \glspl{epic} due to start in the \emph{following} month and to issue an \gls{lcr} on them if necessary.

Note that it is \emph{not possible} to alter history by means of an \gls{lcr}.
That is, if the scheduled start date of an \gls{epic} is already in the past,
it is not possible to move it into the future using a change request. In
this case, there is no option but to carry the variance related to the
late start of the \gls{epic} into the future, to describe that with narratives
(\S\ref{sec:variance-narrative}) where necessary, and to
attempt to address the variance as soon as is possible.

Based on the above, it is clear that technical managers should closely track performance relative to the plan throughout the \gls{cycle}, and proactively file change requests to avoid running variances wherever possible.

\subsection{Closing the Cycle}
\label{sec:cycle-close}

Assuming everything has gone to plan, by the end of a \gls{cycle} all deliverables should be verified and the corresponding \glspl{epic} should be marked as \texttt{done}.
Marking an \gls{epic} as \texttt{done} asserts that the concrete deliverable associated with the \gls{epic} has been provided.
The total cost of that functionality---the \gls{bcws}, calculated as per \S\ref{sec:sps-to-bcws}---is now claimed as value earned.

Epics which are in progress at the end of the \gls{cycle} cannot be closed until they have been completed.
These \glspl{epic} will spill over into the subsequent \gls{cycle}.
It is \emph{not} appropriate to close an in-progress \gls{epic} with a concrete deliverable until that deliverable has been achieved: instead, a variance will be shown until the \gls{epic} can be closed.
Obviously, this will impact the labor available for other activities in the next \gls{cycle}.
(This does not apply to bucket \glspl{epic} (\S\ref{sec:bucket}), which are, by their nature, \gls{timebox}ed within the \gls{cycle}).

Similar logic applies to \glspl{epic} which \emph{have not been started}: if
the planned start date is in the past, they can no longer be rescheduled
by means of an \gls{lcr} (\S\ref{sec:cycle-change}). They must
be completed at the earliest possible opportunity; you will show a
variance until this has been done.

\section{Execution}
\label{sec:execution}

Having defined defined the plan for a \gls{cycle} following \S\ref{sec:cycle-plan}, we execute it by means of a series of month-long sprints.
In this section, we detail the procedures teams are expected to follow during the cycle.

\subsection{Defining Stories}
\label{sec:defining-stories}

Epics have already been defined as part of the \gls{cycle} plan (see \S\ref{sec:planning-epics}).
However, the \gls{epic} is not at an appropriate level for scheduling day-to-day work.
Rather, each \gls{epic} is broken down into a series of self-contained ``stories''.
A \gls{story} describes a planned activity worth between a small fraction of a SP and several \glspl{sp} (more than about 10 is likely an indication that the \gls{story} has not been sufficiently refined).
It must be possible to schedule a \gls{story} within a single sprint, so no \gls{story} should ever be allocated more than 26 \glspl{sp}.

The process for breaking \glspl{epic} down into stories is not mandated. In
some circumstances, it may be appropriate for the technical manager to
provide a breakdown; in others, they may request input from the
developer who is actually going to be doing the work, or even hold a
brainstorming session involving the wider team. This is a management
decision.

It is not required to break all \glspl{epic} down into stories before the \gls{cycle} begins: it may be more appropriate to first schedule a few exploratory stories and use them to inform the development of the rest of the \gls{epic}.
However, do break \glspl{epic} down to describe the stories which will be worked in an upcoming sprint (\S\ref{sec:sprinting}) before the sprint starts.
When doing so, you may wish to leave some spare time to handle emergent work (discussed in \S\ref{sec:bugs}).

Note that there is no relationship enforced between the \gls{sp} total estimated for the \gls{epic} and the sum of the \glspl{sp} of its constituent stories.
It is therefore possible to over- or under-load an \gls{epic}.
This will have obvious ramifications for the schedule.
See \S\ref{sec:cycle-value} for a discussion of its impact on earned value.

\subsection{Sprinting}
\label{sec:sprinting}

Each team organizes its work around periods of work called sprints.
A sprint comprises a defined collection of stories which will be addressed over the course of the month.
These stories are not necessarily (indeed, not generally) all drawn from the same \gls{epic}: rather, while \glspl{epic} divide the \gls{cycle} along logical grounds, sprints divide it along the time axes.

Broadly, executing a sprint falls into three stages:

\begin{enumerate}
\item
  Preparation.

  The team assigns the work that will be addressed during the sprint by choosing from the pre-defined stories (\S\ref{sec:defining-stories}).
  Each team member should be assigned a plausible amount of work, based on the per-\gls{story} \gls{sp} estimates and the likely working rate of the developer (see \S\ref{sec:effort}).

  The process by which work is assigned to team members is a local
  management decision: the orthodox approach is to call a team-wide
  meeting and discuss it, but other approaches are possible (one-to-one
  interactions between developers and technical manager, managerial
  fiat, etc).

  Do not overload developers. Take vacations and holidays into account.
  The sprint should describe a plausible amount of work for the time
  available.
\item
  Execution.

  Daily management during the sprint is a local decision. Suggested best
  practice includes holding regular ``standup'' meetings, at which
  developers discuss their current activities and try to resolve
  ``blockers'' which are preventing them from making progress.

  Stories should be executed following the instructions in the
  \href{http://developer.lsst.io/}{Developer Guide} as regards workflow,
  coding standards, review requirements, and so on. It is important to
  ensure that completed stories are marked as \texttt{done}:
  experience suggests that this can easily be forgotten as developers
  rush on to the next challenge, but it is required to enable us to
  properly track earned value as per\S\ref{sec:cycle-value}.

  When completing a \gls{story} we do not change the number of \glspl{sp} assigned to
  it: the \gls{sp} total reflects our initial estimate of the work involved,
  not the total time invested. This makes it possible to review the
  quality of our estimates at the end of the sprint.

  Avoid adding more stories to a sprint in progress unless it is
  unavoidable (for example, the \gls{story} describes a critical bug that must
  be addressed before proceeding). A sprint should always stay current
  and should be up-to-date with reality; if necessary, already scheduled
  stories may be pushed out of a sprint as soon as it is obvious it is
  unrealistic to expect them to be completed.
\item
  Review.

  At the end of the sprint, step back and consider what has been
  achieved. What worked well? What did not? How can these problems be
  avoided for next time? Was your estimate of the amount of work that
  could be finished in the sprint accurate? If not, how can it be
  improved in future? Refer to the
  \href{https://en.wikipedia.org/wiki/Burn_down_chart}{burn-down chart}
  for the sprint, and, if it diverged from the ideal, understand why.

  Again, the form the review takes is a local management decision: it
  may involve all team members, or just a few.
\end{enumerate}

We use \gls{jira}'s
\href{https://www.atlassian.com/software/jira/agile}{Agile} capabilities
to manage our sprints. Each technical manager is responsible for
defining and maintaining their own agile board. The board may be
configured for either
\href{https://en.wikipedia.org/wiki/Scrum_(software_development)}{Scrum}
or \href{https://en.wikipedia.org/wiki/Kanban_(development)}{Kanban}
style work as appropriate: the former is suitable for planned
development activities (e.g. Science Pipelines development); the latter
for servicing user requests (e.g. providing developer support).

\subsection{Completing Epics}
\label{sec:epic-done}

An \gls{epic} may be marked as \texttt{done} when:

\begin{enumerate}
\item
  It contains at least one completed \gls{story};
\item
  There are no more incomplete \glspl{story} defined within it;
\item
  There are no plans to add more \glspl{story};
\item
  (If applicable, i.e. it is not a bucket, as defined in \S\ref{sec:bucket}) its concrete deliverable has been achieved.
\end{enumerate}

Note that it is not permitted to close an \gls{epic} without defining at least
one \gls{story} within it. Empty \glspl{epic} can never be completed.

When an \gls{epic} is marked as complete, all of its value is earned (\S\ref{sec:cycle-value}).

\subsection{Handling Bugs \& Emergent Work}
\label{sec:bugs}

\subsubsection{Receiving Bug Reports}\label{receiving-bug-reports}

Members of the project who have access to \gls{jira} may report bugs or make feature requests directly using \gls{jira}.
As discussed in \S\ref{sec:jira-maintenance}, technical managers should regularly monitor \gls{jira} for relevant tickets and ensure they are handled appropriately.

Our code repositories are exposed to the world in general through \href{https://github.com/lsst/}{GitHub}.
Each repository on GitHub has a bug tracker associated with it.
Members of the public may report issues or make requests on the GitHub trackers.
Per the \href{https://developer.lsst.io/processes/workflow.html}{Developer Workflow}, all new work must be associated with a \gls{jira} ticket number before it can be committed to the repository.
It is therefore the responsibility of technical managers to file a \gls{jira} ticket corresponding to the GitHub ticket, to keep them synchronized with relevant information, and to ensure that the GitHub ticket is closed when the issue is resolved in \gls{jira}.

The GitHub issue trackers are, in some sense, not a core part of our
workflow, but they are fundamental to community expectations of how they
can interact with the project. Ensure that issues reported on GitHub are
serviced promptly.

In some cases, the technical manager responsible for a given repository
is obvious, and they can be expected to take the lead on handling
tickets. Often, this is not the case: repositories regularly span team
boundaries. Work together to ensure that all tickets are handled.

\subsubsection{Issue Types}\label{issue-types}

We have previously referred to day-to-day work being described by means of stories.
However, \gls{jira} provides us with two additional issue types: ``bug'' and ``improvement''.
Per RFC-43, the semantics of the various issue types are:

\begin{itemize}
\item
  A \gls{story} is the result of breaking down an \gls{epic} into workable units;
\item
  A bug describes a fault or error in code which has already been accepted to master;
\item
  An improvement describes a feature request or enhancement which has not been derived by breaking down the long term plan (i.e., it is an ad-hoc developer or user request).
\end{itemize}

The three issue types are functionally equivalent: these semantic distinctions are for convenience only, and are not rigorously enforced.

In particular, note that all issue types are equivalent in terms of the data which is loaded to the \gls{pmcs}: it makes no distinction between them.
Marking a bug or improvement as \texttt{done} has exactly the same impact on the global earned value state as would completing an equivalent \gls{story}.

\subsubsection{Scheduling}\label{scheduling}

In some cases, a ticket may describe emergent work which must be addressed immediately by adding it to a bucket \gls{epic} (\S\ref{sec:bucket}).
In other cases, it can be deferred to a later \gls{cycle}, or, after appropriate discussion, may be regarded as inappropriate (and can be tagged as \texttt{invalid} or \texttt{won't fix}).
This is a management decision.
When closing a ticket as inappropriate, please take a moment to describe why---the individual who reported it will appreciate an explanation of why it has been rejected, and it will serve as a useful reference the next time somebody suggests the same thing.

A special case of inappropriate tickets are those that duplicate work which has already been described elsewhere.
Please close these as \texttt{invalid}, and add a \gls{jira} link of type \texttt{duplicates} to the original ticket.

Tickets which are obviously filed by mistake may simply be deleted
rather than setting a special status. Please only do this when you are
sure there is no value to leaving an audit trail, and when you have
verified that the original author of the ticket is aware of and
understands the outcome.

\subsubsection{Relationship to Earned
Value}\label{relationship-to-earned-value}

We adopt the position that bugs are a natural part of the software
lifecycle, and hence addressing them at an appropriate level earns value
in the same way as new software development. That is, \glspl{sp} earned by
working on bugs and completing bucket \glspl{epic} contribute to earned value
in the same way as other work.

However, bugs do serve as an bellwether for software quality issues.
It would obviously be inappropriate---and a severe source of schedule \gls{risk}---for the value earned from addressing bugs in existing software to dominate the productivity of the team at the expense of new development.
We expect that no more than around 30\% of schedulable developer time will be dedicated addressing bugs and performing maintenance: any more than this must be carefully justified.

\subsection{Earning Value}
\label{sec:cycle-value}

The basic procedure for earning value during the \gls{cycle} is akin to that discussed in \S\ref{sec:long-term-value} for long term planning.

In short, as we saw in \S\ref{sec:sps-to-bcws}, the \gls{bcws} for a particular \gls{epic} is defined by its \emph{estimated} (i.e.  attached to the \gls{epic} before work commences) SP total and its staff assignment.
When an \gls{epic} is marked as complete (following the criteria in \S\ref{sec:epic-done}, this is the value that is earned.

The \gls{bcwp} for an \gls{epic} is calculated based on the fractional completeness of an \gls{epic}. That is, if an \gls{epic} has a total SP count of \(X\), and the total of stories marked as complete within it is \(Y\), then \(\mathrm{BCWP} = \mathrm{BCWS} \times Y / X\).

Be aware that stories that marked as \texttt{invalid} or \texttt{won't fix} in \gls{jira} are not included in this calculation: they earn no value.

As we saw in \S\ref{sec:defining-stories}, it is not required that the total \glspl{sp} of all the stories contained within an \gls{epic} (the ``planned \glspl{sp}'') is equal to the total SP estimate of the \gls{epic} itself (``estimated \glspl{sp}''). Further, it is permitted to add stories to (or, indeed, remove stories from) the \gls{epic} during the \gls{cycle}. In these cases, we hold to two basic tenets:

\begin{enumerate}
\item
  No \gls{epic} can ever be more than 100\% complete;
\item
  Completeness cannot decrease.
  That is, if an \gls{epic} has been registered as 90\% complete, adding more stories cannot make it \emph{less} complete than before.
\end{enumerate}

In order to meet these criteria, the relative weights of stories will be automatically adjusted on ingest to the \gls{pmcs}.
The detailed algorithm by which this adjustment is made is not publicly documented.

\subsection{JIRA Maintenance}
\label{sec:jira-maintenance}

At any time, new tickets may be added to \gls{jira} by team members.
Please remind your team of the best practice in this respect (RFC-147).
It is the responsibility of technical managers to ensure that new tickets are handled appropriately, updating the schedule to include them where necessary.
It is required that the \texttt{Team} field be set to the appropriate team (RFC-145).
Please regularly monitor \gls{jira} for incomplete tickets and update them appropriately.
Where tickets describe bugs or other urgent emergent work which cannot be deferred, refer to \S\ref{sec:bugs}.

\subsection{Coordination Standup}\label{coordination-standup}

The meeting URL is not included here since this note is publicly
available. Contact the Project Manager for details.

The technical managers meet with the
Project Manager (\S\ref{sec:contacts}) and interested
others (it is not a closed meeting) twice every week. This is a forum to
discuss general project management issues, but, in particular, to
resolve issues which cut across team boundaries and are relevant for the
ongoing sprint.

Meetings take place using \href{https://hangouts.google.com/}{Google
Hangouts} at a pre-arranged URL. Meetings take place at 11:00 (11 a.m.)
Project (Pacific) Time on Tuesdays and Fridays.

\subsection{Monthly Progress Narratives}
\label{sec:monthly-narrative}

Every calendar month, each technical manager is required to support the
Project Manager with a report on the activities of their group. This
report should be generally submitted no later than tenth of the month
(refer to \S\ref{sec:reporting-cycle}), but this may be moved earlier on
occasion. You are encouraged to submit your report as early in the month
as possible.

Submit your report by editing the
\href{https://drive.google.com/drive/u/0/folders/0BxgFbTQURmr6TUJleXZaY2ZNcEE}{template
for the appropriate month} on Google Docs. You need to fill in all the
sections with your name attached; when complete, remove your name.
Provide a brief (one or two sentences) high level summary, a per-\gls{wbs}
breakdown of work over the month being reported on and plans for the
upcoming month, as well as describing any recruitment activities
(positions opened, interviews conducted, appointments made, etc). Refer
to previous reports for examples of the style used (but note that they
are not not always consistent).

\section{Reporting Actuals}
\label{sec:actuals}

In order to comply with the earned value management system (\S\ref{sec:evms}), it is necessary to track the actual cost of work being performed (the ``actuals'') in each leaf \gls{element} of the \gls{wbs}.
That is, whenever an invoice is issued from a subcontracting institution to AURA, it must be broken down into dollar charges against individual \gls{wbs} elements.

Some institutions rigorously track how staff are spending their time
(e.g. by filling in timesheets), and may directly make that information
available to AURA as part of the invoicing process. In this case, the
technical manager need take no further action.

Other institutions do not rigorously check staff activity and/or do not supply this information to AURA when invoicing.
In this case, the technical manager is responsible for breaking down the invoice by \gls{wbs} and forwarding that to the relevant AURA contracts officer (check with the Project Manager (\S\ref{sec:contacts}) if you are unsure who that is).
Note that, since \glspl{sp} reflect estimated, not actual, time spent on work (\S\ref{sec:sprinting}), it is \emph{not} appropriate to simply allocate actual costs based on SP totals.

Typically, expenses are accrued at a broadly constant rate for each individual: salaries do not vary much from month to month.
However, in some months, a given developer may be significantly less productive than others (for example, due to paid vacation).
In these cases, it is appropriate to spread the cost across all the \gls{wbs} \glspl{element} the developer has been working on.

A typical invoice breakout should be supplied in a spreadsheet similar to that shown in Table \ref{tab:invoice}.

\begin{table}
\begin{longtable}[]{@{}llllll@{}}
\hline
\multicolumn{3}{c}{Invoice Date: YYYY-MM-DD} & \multicolumn{3}{c}{Period: YYYY-MM-DD/DD} \\
\hline
Total     & 02C.0N.00  & 02C.0N.01 & 02C.0N.02 & ... & 02C.0N.0M \\
          & KLM20N00A  & KLM20N01A & KLM20N02A & ... & KLM20N0MA \\
\$ABCD.EF & \$GHIJ.KL  & \$MNOP.QR & \$STUV.WX & ... & \$YZ.00 \\
\hline
\end{longtable}
\caption{Example invoice breakout showing dollar values allocated to both WBS elements and corresponding account numbers.}
\label{tab:invoice}
\end{table}

Note that when reporting actuals at this level it is not required to
provide a mapping from dollar values to individuals who did the work.
However, it is important to note that, should the Project be audited in
the future, it is perfectly possible that they will wish to examine such
a mapping. You should therefore keep records which will enable you to
provide it upon request.

\section{Standard Reporting Cycle}
\label{sec:reporting-cycle}

\begin{itemize}
\item
  During the first week of the calendar month, data from \gls{jira} together with actual costs (labor charges, etc) are ingested to the \gls{pmcs} system.
  This indicates the progress of all activities and shows any Earned Value variances.
  This information is made available to technical managers through \gls{ecam}.
\item
  During the second week of the calendar month:

  \begin{quote}
  \begin{itemize}
  \item
    Variance narratives (\S\ref{sec:variance-narrative}), where necessary, must be submitted through \gls{ecam}.
  \item
    The monthly progress narrative (\S\ref{sec:monthly-narrative}) must be submitted through Google Docs by the tenth day of the month.
  \end{itemize}
  \end{quote}
\item
  The DM Project Manager assembles extended and summary reports, based
  on the reports received from the institutions. The extended report is
  periodically examined by Federal auditors, while the summary report is
  provided to senior management and the AMCL for review.
\end{itemize}

\section{Staffing Changes}\label{staffing-changes}

In addition to onboarding procedures at your local institution, please
be aware of

\begin{itemize}
\item
  The LSST \href{https://project.lsst.org/onboarding}{New Employee
  Onboarding} material, and
\item
  The DM
  \href{https://developer.lsst.io/getting-started/onboarding.html}{Developer
  Onboarding Checklist}
\end{itemize}

and direct new recruits to them when they join your team.

We maintain a
\href{https://docs.google.com/spreadsheets/d/1G9KXBJJHfWkVDQeApfXaN_nZjD_YUJlHiEDOzhTy-0c/edit?usp=drive_web}{spreadsheet}
listing all members of the DM team. Ensure it is kept up to date with
the current and projected staffing within your team.
