\newacronym[description={Actual Cost of Work Performed (often referred to as ``actuals'')}]{acwp}{ACWP}{Actual Cost of Work Performed}

\newacronym{bcwp}{BCWP}{Budgeted Cost of Work Performed}

\newacronym{bcws}{BCWS}{Budgeted Cost of Work Scheduled}

\newglossaryentry{unit}
{
 name={budgeted (labor) unit},
 description={An hour of work}
}

\newacronym
 [description={Control Account Manager.
               A CAM is responsible for the scope, schedule and budget for one or more \glspl{ca}}]
 {cam}{CAM}{Control Account Manager}

\newacronym
 [description={Change Control Board.
               All changes to the baselined plan must be approved by the CCB.
               See \citeds{LPM-19} for details}]
 {ccb}{CCB}{Change Control Board}

\newglossaryentry{ca}
{
 name={control account},
 description={An intersection point between the \gls{wbs} and the \gls{obs}.
              For example, work performed at IPAC (1.03) on the Science User Interface (1.02C.05) is managed by a single control account}
}

\newacronym
 [description={Cost Performance Index. Defined as \gls{bcwp} $\div$ \gls{acwp}}]
 {cpi}{CPI}{Cost Performance Index}

\newacronym
 [description={Cost Variance. Defined as \gls{bcwp} $-$ \gls{acwp}}]
 {cv}{CV}{Cost Variance}

\newglossaryentry{cycle}
{
 name=cycle,
 description={The time period over which detailed, short-term plans are defined and executed.
              Normally, cycles run for six months, and culminate in a new release of the LSST Software Stack, however this need not always be the case}
}

\newglossaryentry{ecam}
{
 name=eCAM,
 description={The \href{https://msweb.lsstcorp.org/eCAM/}{eCAM Notebook}, a tool which reports information from the \gls{pmcs}.
              It provides a convenient view of the current status of the project in terms of \gls{evms}}
}

\newglossaryentry{element}
{
 name=element,
 description={A node in the hierarchical project \gls{wbs}}
}

\newglossaryentry{epic}
{
 name=epic,
 description={A self contained work with a concrete deliverable which my be scheduled to take place with a single \gls{cycle} and \gls{wbs} \gls{element}}
}

\newacronym
 [description={Earned Value Management System. See the brief description in \S\ref{sec:evms}, or refer to formal training}]
 {evms}{EVMS}{Earned Value Management System}

\newglossaryentry{jira}
{
 name=JIRA,
 description={Issue and project tracking software produced by \href{https://www.atlassian.com}{Atlassian}.
              \href{https://jira.lsstcorp.org}{LSST's JIRA} is a core interface between technical managers, their teams, and the \gls{pmcs}}
}

\newacronym
 [description={LSST Change Request.
               It is necessary to submit a change request to alter any ``baselined'' aspect of the project.
               This includes, for example, altering change controlled plans, or epics that have been loaded to the \gls{pmcs}}]
 {lcr}{LCR}{LSST Change Request}

\newacronym
 [description={
  Level of Effort.
  LOE work is that which does not correspond to a specific deliverable.
  A detailed definition is provided in \citeds{LDM-472}; see also the discussion in \S\ref{sec:loe}}]
 {loe}{LOE}{Level of Effort}

\newacronym
 [description={Major Research Equipment and Facilities Construction.
               The terms under which LSST's \gls{nsf} funding has been issued; we are required to strictly adhere to them}]
 {mrefc}{MREFC}{Major Research Equipment and Facilities Construction}

\newacronym{nsf}{NSF}{National Science Foundation}

\newacronym{obs}{OBS}{Organizational Breakdown Structure}

\newglossaryentry{risk}
{
 name=risk,
 description={Risks are (per ISO 31000) ``the effect of uncertainty upon objectives''.
              For the purposes of this document, that corresponds to the impact of unplanned or unpredictable events upon the cost or schedule of the Project.
              The Project maintains a register of risks, which includes probability estimates and possible mitigations}
}

\newacronym
 [description={
  Project Management Control System.
  The PMCS is not a single piece of software, but rather an interlocking suite of tools.
  In general, the CAM need not interact with PMCS directly, but only through the eCAM and JIRA tools: it is safe to treat PMCS as a ``black box''.
  Occasionally, individual PMCS components such as Primavera or Deltek Cobra escape this abstraction and appear in documentation}]
 {pmcs}{PMCS}{Project Management Control System}

\newglossaryentry{resloading}
{
 name={resource loading},
 description={Assigning particular resources (in software development, almost always staffing) to particular tasks.
              A ``resource loaded plan'' provides a mapping of resources to the plan throughout execution}
}

\newacronym
 [description={Story Point.
               Used to estimate the duration of tasks in JIRA.
               One SP is equivalent to 4 hours of uninterrupted effort by a competent developer}]
 {sp}{SP}{Story Point}

\newacronym
 [description={Schedule Performance Index. Defined as \gls{bcwp} $\div$ \gls{bcws}}]
 {spi}{SPI}{Schedule Performance Index}

\newglossaryentry{sprint}
{
 name=sprint,
 description={A defined period of work for a particular team.
               Typically, sprints are one calendar month long, but this is not required}
}

\newacronym
 [description={Science Quality and Reliability Engineering.
               One of the teams which makes up the Data Management Group.
               SQuaRE coordinates the end-of-cycle release of the codebase (refer to \S\ref{sec:cycle-cadence}), and therefore plays a pivotal role in planning}]
 {square}{SQuaRE}{Science Quality and Reliability Engineering}

\newglossaryentry{story}
{
 name=story,
 description={A \gls{jira} issue type describing a scheduled, self-contained task worked as part of an \gls{epic}.
              Typically, stories are appropriate for work worth between a fraction of a \gls{sp} and 10 \glspl{sp}; beyond that, the work is insufficiently fine-grained to schedule as a story.
              While fractional \glspl{sp} are fine, all stories involve work, so the \glspl{sp} total of an in progress or completed story should not be 0},
 plural=stories
}

\newacronym
 [description={Schedule Variance. Defined as \gls{bcwp} $-$ \gls{bcws}}]
 {sv}{SV}{Schedule Variance}


\newglossaryentry{timebox}
{
 name=timebox,
 description={A limited time period assigned to a piece of work or other activity.
              Useful in scheduling work which is not otherwise easily limited in scope, for example research projects or servicing user requests}
}

\newacronym{wbs}{WBS}{Work Breakdown Structure}
